\documentclass[12pt,a4paper]{article}
\usepackage[utf8]{inputenc}
\usepackage[russian]{babel}
\usepackage{enumitem}
\usepackage{amsmath}
\usepackage{amsfonts}
\usepackage{amssymb}
\usepackage{amsthm}
\usepackage{booktabs}
\usepackage{graphicx}
\newtheorem{theorem}{Теорема}


\title{Технический отчет по криптоанализу криптосистемы}
\author{}
\date{\today}

\begin{document}

\maketitle

\section{Описание криптосистемы}
Состояние шифра разделено на 4 ниббла по 4 бита каждый: $X = (x_0, x_1, x_2, x_3)$.

\subsection{Параметры}
\begin{itemize}
    \item \textbf{Размер блока:} 16 бит.
    \item \textbf{Количество раундов:} $R = 6$.
    \item \textbf{S-блок (G):} \{13, 6, 0, 10, 15, 7, 14, 11, 9, 1, 5, 3, 4, 12, 8, 2\}.
    \item \textbf{Функция раунда:} $F(x_2, x_3, k) = G(x_2 \oplus G(k \oplus x_3))$.
    \item \textbf{Расписание ключей:} Используются 4 базовых ключа $\Theta = (\theta_1, \theta_2, \theta_3, \theta_4)$. Раундовые ключи формируются как $k_1=\theta_1, k_2=\theta_2, k_3=\theta_3, k_4=\theta_4, k_5=\theta_1, k_6=\theta_2$.
\end{itemize}

\subsection{Алгоритм зашифрования}
На каждом раунде $i \in \{1, \dots, 6\}$ выполняется преобразование:
\begin{align*}
    temp &= x_0 \oplus F(x_2, x_3, k_i) \\
    x_0 &\leftarrow x_1 \\
    x_1 &\leftarrow x_2 \\
    x_2 &\leftarrow x_3 \\
    x_3 &\leftarrow temp
\end{align*}
Данная операция соответствует циклическому сдвигу влево с обновлением младшего ниббла.

\section{Функция расшифрования}
Для восстановления состояния до раунда необходимо инвертировать операцию сдвига и XOR. Пусть текущее состояние (после раунда) равно $Y = (y_0, y_1, y_2, y_3)$. Тогда состояние до раунда $X = (x_0, x_1, x_2, x_3)$ вычисляется следующим образом:

\begin{enumerate}
    \item Инверсия сдвига:
    \begin{align*}
        x_1 &= y_0 \\
        x_2 &= y_1 \\
        x_3 &= y_2
    \end{align*}
    \item Обратный XOR для восстановления $x_0$:
    \begin{equation}
        x_0 = y_3 \oplus F(x_2, x_3, k) = y_3 \oplus F(y_1, y_2, k)
    \end{equation}
\end{enumerate}

Полная формула функции расшифрования одного раунда:
\begin{equation}
    \text{Dec}(y_0, y_1, y_2, y_3, k) = (y_3 \oplus G(y_1 \oplus G(k \oplus y_2)), y_0, y_1, y_2)
\end{equation}

\section{Методология атак и экспериментальные результаты}

\subsection{Дифференциальный криптоанализ}
Атака основана на поиске итеративного дифференциала для 5 раундов.

\subsubsection{Аналитический поиск характеристик (DDT)}
Вместо стохастического поиска был применен аналитический метод на основе Таблицы Распределения Разностей (DDT). Был разработан алгоритм поиска в ширину (Beam Search), который выявил следующую оптимальную траекторию:

\begin{itemize}
    \item Входная разность: $\Delta X = (10, 8, 12, 0)$
    \item Выходная разность (5 раундов): $\Delta Y = (0, 0, 0, 2)$
    \item Теоретическая вероятность: $P_{theory} \approx 0.0439$
\end{itemize}

\subsubsection{Проведение атаки}
Для проверки теоретической модели было сгенерировано 100,000 пар текстов с фиксированной разностью $\Delta X$. Ожидаемое количество "полезных" пар составляло $100000 \times 0.0439 \approx 4390$.

\subsection{Линейный криптоанализ}
\subsubsection{Поиск линейных аппромаций}
Был реализован алгоритм поиска линейных смещений (biases). Лучшая найденная аппроксимация для 5 раундов:
\begin{itemize}
    \item Входная маска: $\alpha = 0x0004$
    \item Выходная маска: $\beta = 0x2140$
    \item Теоретическое смещение: $\epsilon \approx 0.1259$
\end{itemize}

\section{Выводы и результаты работы программы}

В данном разделе представлены итоговые результаты работы программного комплекса, подтверждающие успешность проведенных атак.

\subsection{Результаты дифференциальной атаки}
В ходе атаки на ключ 6-го раунда были получены следующие результаты (количество совпадений Hits):

\begin{enumerate}
    \item \textbf{Key 3 (0x3):} 4442 совпадения.
    \item \textbf{Key 1 (0x1):} 4442 совпадения.
    \item Ближайший "шумовой" кандидат (Key 5): 1085 совпадений.
\end{enumerate}

\textbf{Анализ:} Экспериментальное значение (4442) практически идеально совпало с теоретическим предсказанием (4390). Это подтверждает корректность построенной аналитической модели. Истинный ключ $k_6=3$ был успешно выявлен (наряду с ложным кандидатом 1, что требует дополнительной проверки на других парах, но не отменяет факт взлома).

\subsection{Результаты линейной атаки}
Линейная атака на 50,000 парах показала следующие значения смещения (Bias):

\begin{enumerate}
    \item \textbf{Key 3 (0x3):} $Bias \approx 0.1223$.
    \item Ближайший конкурент (Key 5): $Bias \approx 0.0264$.
\end{enumerate}

\textbf{Анализ:} Экспериментальное смещение (0.1223) очень близко к теоретическому пределу (0.1259). Отрыв от шума составляет 4.6 раза. Линейная атака показала более высокую селективность, однозначно указав на правильный ключ без "призраков".

\section{Заключение}
Криптосистема демонстрирует низкую стойкость к классическим методам анализа.
\begin{itemize}
    \item \textbf{Дифференциальный метод:} Позволил сузить пространство ключа до 2 вариантов с использованием 100,000 пар.
    \item \textbf{Линейный метод:} Позволил однозначно восстановить ключ с использованием 50,000 пар.
\end{itemize}
Истинный ключ $k_6=3$ (соответствующий $\theta_2$ из расписания) был успешно восстановлен обоими методами.


\end{document}